% arXiv-compatible article template
% Compile with pdfLaTeX + BibTeX
% Uses arxiv.sty from https://github.com/kourgeorge/arxiv-style

\documentclass[11pt]{article}

% ============================================================
% ARXIV STYLE
% ============================================================
\usepackage{arxiv}

% ============================================================
% PACKAGES
% ============================================================

% --- Typography ---
\usepackage[T1]{fontenc}        % Type 1 fonts (required for arXiv)
\usepackage{microtype}          % Improved typography
\usepackage{xspace}             % Smart spacing after macros

% --- Math ---
\usepackage{amsmath, amssymb, mathtools}

% --- Colors ---
\usepackage[dvipsnames]{xcolor}

% --- Figures and Tables ---
\usepackage{graphicx}
\usepackage{booktabs}
\usepackage{float}
\usepackage{caption}
\usepackage{subcaption}

% --- Lists ---
\usepackage{enumitem}

% --- Bibliography ---
\usepackage[numbers, sort&compress]{natbib}

% --- Hyperlinks (load last) ---
\usepackage[
    colorlinks=true,
    linkcolor=blue!60!black,
    citecolor=green!50!black,
    urlcolor=blue!70!black
]{hyperref}
\usepackage[capitalise, noabbrev]{cleveref}

% ============================================================
% CUSTOM COMMANDS
% ============================================================

% Math shortcuts
\newcommand{\R}{\mathbb{R}}
\newcommand{\N}{\mathbb{N}}
\newcommand{\E}{\mathbb{E}}
\newcommand{\Var}{\mathrm{Var}}

% Text shortcuts
\newcommand{\ie}{i.e.\xspace}
\newcommand{\eg}{e.g.\xspace}
\newcommand{\cf}{cf.\xspace}
\newcommand{\etal}{et~al.\xspace}

% TODO notes (disable for final version)
\usepackage{todonotes}
% \usepackage[disable]{todonotes}  % Uncomment to hide all todos

% ============================================================
% METADATA
% ============================================================

\title{Your Paper Title Here}
\shorttitle{Short Title for Header}

\author{
    First Author\thanks{Corresponding author: \texttt{first@university.edu}} \\
    \small Department of Computer Science \\
    \small University Name \\
    \and
    Second Author \\
    \small Research Lab \\
    \small Institution Name
}

\date{\today}

% ============================================================
% DOCUMENT
% ============================================================
\begin{document}

\maketitle

% ------------------------------------------------------------
\begin{abstract}
    Brief summary of the paper (150--250 words). State the problem, 
    your approach, main results, and conclusions.
\end{abstract}

\noindent\textbf{Keywords:} keyword1, keyword2, keyword3

% ------------------------------------------------------------
\section{Introduction}
\label{sec:introduction}

Introduce the problem and motivation. Prior work has shown promising 
results~\citep{example2023}. We build upon these findings and propose 
a novel approach.

\todo[inline]{Add more background references}

Our main contributions are:
\begin{itemize}[noitemsep]
    \item First contribution
    \item Second contribution
    \item Third contribution
\end{itemize}

% ------------------------------------------------------------
\section{Related Work}
\label{sec:related}

Discuss relevant prior work. \citet{example2023} introduced the 
foundational concepts. See also~\citep{anotherwork2022} for a 
comprehensive survey.

% ------------------------------------------------------------
\section{Method}
\label{sec:method}

Describe your approach in detail.

\subsection{Problem Formulation}

Let $x \in \R^n$ denote the input. We aim to find:
\begin{equation}
    \min_{w} \frac{1}{N} \sum_{i=1}^{N} \mathcal{L}(f(x_i; w), y_i)
    \label{eq:objective}
\end{equation}
where $\mathcal{L}$ is the loss function.

\subsection{Algorithm}

Our algorithm proceeds as follows (see \cref{eq:objective}):
\begin{enumerate}[noitemsep]
    \item Initialize parameters $w_0$
    \item For each iteration $t = 1, \ldots, T$:
    \begin{enumerate}
        \item Compute gradient $\nabla \mathcal{L}$
        \item Update $w_{t+1} = w_t - \eta \nabla \mathcal{L}$
    \end{enumerate}
    \item Return $w_T$
\end{enumerate}

% ------------------------------------------------------------
\section{Experiments}
\label{sec:experiments}

\subsection{Setup}

Describe datasets, baselines, and evaluation metrics.

\subsection{Results}

Present your results. See \cref{tab:results} and \cref{fig:example}.

\begin{table}[htbp]
    \centering
    \caption{Comparison with baselines on benchmark dataset.}
    \label{tab:results}
    \begin{tabular}{lcc}
        \toprule
        Method & Accuracy (\%) & Runtime (s) \\
        \midrule
        Baseline 1 & 85.2 & 12.3 \\
        Baseline 2 & 87.1 & 15.7 \\
        \textbf{Ours} & \textbf{91.4} & 14.2 \\
        \bottomrule
    \end{tabular}
\end{table}

\begin{figure}[htbp]
    \centering
    % \includegraphics[width=0.8\linewidth]{figures/results.pdf}
    \fbox{\parbox{0.6\linewidth}{\centering [Figure placeholder]}}
    \caption{Performance comparison across different settings.}
    \label{fig:example}
\end{figure}

% ------------------------------------------------------------
\section{Discussion}
\label{sec:discussion}

Interpret results, discuss limitations, and suggest future work.

% ------------------------------------------------------------
\section{Conclusion}
\label{sec:conclusion}

Summarize key findings and contributions.

% ------------------------------------------------------------
\section*{Acknowledgments}

Thank funding sources and collaborators. 
\todo{Add grant numbers}

% ------------------------------------------------------------
\bibliographystyle{plainnat}
\bibliography{references}

% ------------------------------------------------------------
% APPENDIX (optional)
% ------------------------------------------------------------
\appendix
\section{Additional Results}
\label{app:additional}

Include supplementary material here.

\end{document}
